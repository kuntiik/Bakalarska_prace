
\ctusetup{
	preprint = \ctuverlog,
%	mainlanguage = english,
    titlelanguage = czech,
	mainlanguage = czech,
	otherlanguages = {slovak,english},
	title-czech = {Detekce objektů z hloubkové kamery},
	title-english = {Object Detection from Depth camera},
	%subtitle-czech = {Cesta do tajů kdovíčeho},
	%subtitle-english = {Journey to the who-knows-what wondeland},
	doctype = B,
	faculty = F3,
	department-czech = {Katedra kybernetiky},
	department-english = {Department of Cybernetics},
	author = {Lukáš Kunt},
	supervisor = {RNDr. Petr Štěpán, Ph.D.},
	supervisor-address = {Ústav X, \\ Uliční 5, \\ Praha 99},
	%supervisor-specialist = {John Doe},
	fieldofstudy-english = {Mathematical Engineering},
	subfieldofstudy-english = {Mathematical Modelling},
	%fieldofstudy-czech = {Matematcké inženýrství},
	subfieldofstudy-czech = {Kybernetika a robotika},
	keywords-czech = {detekce objektů, stereokamra, konvexní obal, mračno bodů, RGB-D, RANSAC, PCA},
	keywords-english = {object detection, stereo camera, convex hull, point cloud, RGB-D, RANSAC, PCA},
	day = 11,
	month = 5,
	year = 2020,
    specification-file = {zadani.pdf},
    front-specification = true,
%	front-list-of-figures = false,
%	front-list-of-tables = false,
%	monochrome = true,
%	layout-short = true,
}

\ctuprocess

\addto\ctucaptionsczech{%
	\def\supervisorname{Vedoucí}%
	\def\subfieldofstudyname{Studijní program}%
}

%\ctutemplate{include.specification}

\ctutemplateset{maketitle twocolumn default}{
	\begin{twocolumnfrontmatterpage}
		\ctutemplate{twocolumn.thanks}
		\ctutemplate{twocolumn.declaration}
		\ctutemplate{twocolumn.abstract.in.titlelanguage}
		\ctutemplate{twocolumn.abstract.in.secondlanguage}
		\ctutemplate{twocolumn.tableofcontents}
		\ctutemplate{twocolumn.listoffigures}
        \ctutemplate{twocolumn.symb}
	\end{twocolumnfrontmatterpage}
}
\begin{thanks}
Tímto bych chtěl poděkovat panu RNDr. Petru Štěpánovi,  Ph.D. za vedení práce. Zároveň bych  chtěl poděkovat své rodině a přítelkyni za podporu při studiu a tvorbě práce.
\end{thanks}

\begin{abstract-czech}
    Tato práce se zabývá detekcí objektů z hloubkové složky RGB-D dat, jejichž zdrojem je stereokamera Intel\textregistered{} Realsense$^{TM}$. Nejprve je provedena rešerše dostupných metod pro hledání normálového vektoru plochy v mračnech bodů, segmentaci obrazu a detekci objektů. Posléze je těchto znalostí využito k navržení vlastního programu na detekci normálového vektoru plochy a následnému přesnému určení polohy na ploše ležících objektů. Tyto programy vycházejí zejména z RANSAC algoritmu a PCA jsou kombinovány s naši apriorní znalostí tvaru objektů. Navržené programy jsou následně vyhodnoceny na manuálně označených datech, čímž je ověřeno, že jsou schopny přesné detekce v reálném čase.
\end{abstract-czech}
    %Tato práce se zabývá detekcí objektů z RGB-D dat. Zdrojem těchto dat je stereokamera Intel\textregistered{} Realsense$^{TM}$. Nejprve je provedena rešerše dostupných metod pro hledání normálového vektoru plochy v mračnech bodů, segmentaci obrazu a detekci objektů. Posléze je těchto znalostí využito k navření vlastního programu na detekci normálového vektoru plochy a následné detekci na ploše ležícich objektů. Tyto programy vycházejí zejména z RANSAC algoritmu a PCA a kombinují tyto s naši apriorní znalostí tvaru objektů. Tyto programy jsou následně vyhodnoceny na manuálně označených datech,čímž je ověřeno že jsou schopny přesné detekce v reálném čase.

\begin{abstract-english}
    This work deals with object detection from the depth part of RGB-D data, which are produced by stereocamera Intel\textregistered{} Realsense$^{TM}$. Firstly, research of available methods for plane normal estimation from point clouds, picture segmentation and object detection is conducted. Based on this knowledge a program for detection of the surface normal and then precise location of on surface laying objects is proposed. This program is mainly based on RANSAC algorithm and PCA, which are combined with our aprior knowledge of the shapes of objects. Proposed programs are then evaluated on manualy labeled dataset, thereby it is validated, that they are capable of high accuracy object detection in real time.
\end{abstract-english}
\begin{declaration}
Prohlašuji, že jsem předloženou práci
vypracoval samostatně a že jsem uvedl
veškeré použité informační zdroje v
souladu s Metodickým pokynem o
dodržování etických principů při přípravě
vysokoškolských závěrečných prací. \\
V Praze dne 26. května 2017 \\ \\
.................................\\
podpis autora práce
\end{declaration}
%\begin{symb}
    %RANSAC \newline
    %PCA \\
    %ICP \\
    %IMU \\
    %CCL \\
    %DBSCAN \\ \\
    %RGB-D \\
    %MBZIRC \\ \\
    %LIDAR \\
    %HOG \\
    %CAD \\
    %SURF \\
    %SIFT \\
    %BRIEF \\ \\
    %PCL \\ 
    %ORB \\
    %SVM

    %\newpage \noindent
    %Random sample consensus \\
    %Principal Component Analysis \\
    %Iterative Closest Point \\
    %Inerční měřící jednotka\\
    %Connected-component labeling \\
    %Density-based spatial clustering of applications with noise \\ 
    %Red, green, blue - depth \\
    %Mohamed Bin Zayed International Robotics Challenge \\
    %Light Detection and Ranging \\
    %Histogram of Oriented Gradients \\
    %Computer Aided Design \\
    %Speeded up robust features \\
    %Scale-invariant feature transform \\
    %Binary Robust Independent Elementary Features \\
    %Point Cloud Library \\
    %Oriented FAST and rotated BRIEF \\
    %Support vector machines
%\end{symb}

\begin{symb}
    PCA \\ 
    IMU \\ 
    RGB-D \\ 
    RANSAC \newline
    ICP \\
    CCL \\
    DBSCAN \\ \\
    MBZIRC \\ \\
    LIDAR \\
    HOG \\
    CAD \\
    SURF \\
    SIFT \\
    BRIEF \\ \\
    PCL \\ 
    ORB \\
    SVM

    \newpage \noindent
    Principal Component Analysis \\
    %Principal Component Analysis (Analýza hlavních komponent) \\
    Intertial measurement unit \\
    %Intertial measurement unit (Inerční měřící jednotka)\\
    %Red, green, blue - depth  (Červená, zelená, modrá - hloubka)\\
    Red, green, blue - depth  \\
    Random sample consensus \\
    Iterative Closest Point \\
    Connected-component labeling \\
    Density-based spatial clustering of applications with noise \\ 
    Mohamed Bin Zayed International Robotics Challenge \\
    Light Detection and Ranging \\
    Histogram of Oriented Gradients \\
    Computer Aided Design \\
    Speeded up robust features \\
    Scale-invariant feature transform \\
    Binary Robust Independent Elementary Features \\
    Point Cloud Library \\
    Oriented FAST and rotated BRIEF \\
    Support vector machines
\end{symb}
% Theorem declarations, this is the reasonable default, anybody can do what they wish.
% If you prefer theorems in italics rather than slanted, use \theoremstyle{plainit}
%\theoremstyle{plain}
%\newtheorem{theorem}{Theorem}[chapter]
%\newtheorem{corollary}[theorem]{Corollary}
%\newtheorem{lemma}[theorem]{Lemma}
%\newtheorem{proposition}[theorem]{Proposition}

%\theoremstyle{definition}
%\newtheorem{definition}[theorem]{Definition}
%\newtheorem{example}[theorem]{Example}
%\newtheorem{conjecture}[theorem]{Conjecture}

%\theoremstyle{note}
%\newtheorem*{remark*}{Remark}
%\newtheorem{remark}[theorem]{Remark}

%\setlength{\parskip}{5ex plus 0.2ex minus 0.2ex}
